\documentclass[12pt]{article}
\usepackage{times}
\usepackage{blindtext}
\usepackage{url}

\title{Mathematical Observation on the Homelessness Crisis through Mathematical Methods}
\author{Mufaro Machaya, Nam Le, Ben Stillwell \\ Cy-Fair Senior High School \\ Cypress, Texas}
\date{March 3, 2024}


\begin{document}

\newpage

\large{Executive Summary}
\blindtext

\maketitle
\newpage
\tableofcontents

\newpage

\section{Introduction}
Understanding the housing shortage has become more important than ever, as rates of homelessness have reached
unprecedented and potentially dire level\cite{Census2010ACSDP1Y2010.DP04}. Moving into the future, it is
undeniably critical for promoting the general welfare of United States populations to best understand housing for
influencing public policy across major cities in the United States. Thus, we have prepared the following mathematical
models to best understand such a trend.

\newpage

\section{It was the Best of Times}

\subsection{Restatement of the Problem}
The first problem asks us to develop a mathematical model to predict changes to the housing supply over the next 50
years in two cities of our choosing: Seattle, Washington; and Albequerque, New Mexico.

\subsection{Assumptions and Justifications}
\subsection{Model Development}
\subsection{Results}
\subsection{Reflecting on the Model}

\newpage

\section{It was the Worst of Times}

\subsection{Restatement of the Problem}
\subsection{Assumptions and Justifications}
\subsection{Model Development}
\subsection{Results}
\subsection{Reflecting on the Model}

\newpage

\section{Rising from this Abyss}

\subsection{Restatement of the Problem}
\subsection{Assumptions and Justifications}
\subsection{Model Development}
\subsection{Results}
\subsection{Reflecting on the Model}

\newpage

\bibliographystyle{plain}
\bibliography{sources}

\end{document}
