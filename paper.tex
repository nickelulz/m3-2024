\documentclass[12pt]{article}
\usepackage{times}
\usepackage{blindtext}
\usepackage{url}
\usepackage{float}

\title{Observing the United States' Homelessness Crisis through Mathematical Methods}
\author{Mufaro Machaya, Nam Le, Ben Stillwell \\ Cy-Fair Senior High School \\ Cypress, Texas}
\date{March 3, 2024}


\begin{document}

\newpage

\section*{Executive Summary}
\blindtext

\maketitle
\newpage
\tableofcontents

\newpage

\section{Introduction}
Understanding the housing shortage has become more important than ever, as rates of homelessness have reached
unprecedented and potentially dire level\cite{NPR-ABGC-2022}. Moving into the future, it is
undeniably critical for promoting the general welfare of United States populations to best understand housing for
influencing public policy across major cities in the United States. Thus, we have prepared the following mathematical
models to best understand such a trend.

\newpage

\section{It was the Best of Times}

\subsection{Restatement of the Problem}
The first problem asks us to develop a mathematical model to predict changes to the housing supply over the next 50
years in two cities of our choosing: Seattle, Washington; and Albequerque, New Mexico.

\subsection{Assumptions and Justifications}

\textit{There will be no immense, drastic, and long-term permanent changes to housing production possibility
within the next fifty years.} \\

\noindent
Justification: Events that will drastically change the ability to produce housing, such as immense wars or widespread
natural disasters, are neigh-impossible to detect accurately with a simple mathematical model. The only events that can
be accurately tracked through mathematical means over short periods of time are cyclical events, like economic
recessions or inflation. \\

\noindent
\textit{Occupied housing will not be considered part of the available total housing supply.} \\

\noindent
Justification: In regards to fixing the homelessness crisis, housing that can be considered available for allocation to
unhoused people inherently must be unoccupied by others. \\

\noindent
\textit{Differences in median housing costs and median income per capita across regions in both Albequerque and Seattle
will not be considered in determining housing availability.} \\

\noindent
Justification: These factors hold a strong correlation with housing unavailability, yet we assume them to have no
causal relationship. \\

\noindent
\textit{The National GDP will have a strong-enough correlation with housing availability to aid in accurately
accounting for cyclical economic trends.} \\

\noindent
Justification: We fully recognize that national economic health will not be inherently causal with housing availability
in select cities, but on a macroeconomic scale, the strength of the economy will be reasonably provide a benchmark for
predicting the economic decisions of housing construction companies and consumers.
This relies on the assumption that in times of economic prosperity,

\subsection{Model Development}
For our housing availability model, we choose to use a simple vector machine that regresses historical housing availiability
data for both Albequerque and Seattle, and from there, we cross-reference this data with historical economic trends
across the United States to best model housing availability over time given cyclical economic trends.

\noindent
To begin, we trained the model on the following datasets for housing availability in Albequerque and Seattle and the
national GDP of the United States.

\begin{table}[H]
  \centering
  \begin{tabular}{|c c c c|} 
    \hline 
    Year & Total housing units & Occupied units & Vacant units \\ [0.5ex]
    \hline
    2010  &  234,891 & 217,256 & 17,635 \\
    2011  &  237,735 & 220,060 & 17,675 \\
    2012  &  239,718 & 222,584 & 17,134 \\
    2013  &  240,277 & 222,491 & 17,786 \\
    2014  &  240,961 & 222,868 & 18,093 \\
    2015  &  241,326 & 222,098 & 19,228 \\
    2016  &  242,070 & 221,320 & 20,750 \\
    2017  &  243,402 & 221,119 & 22,283 \\
    2018  &  244,382 & 222,748 & 21,634 \\
    2019  &  245,476 & 224,166 & 21,310 \\
    2020  &  247,926 & 229,701 & 18,225 \\
    2021  &  252,924 & 236,191 & 16,733 \\
    2022  &  255,178 & 239,800 & 15,378 \\ [1ex] 
    \hline
  \end{tabular}
  \caption{Housing Statistics in Albequerque, New Mexico \cite{Census2010ACSDP1Y2010.DP04}}
\end{table}

\begin{table}[H]
  \centering
  \begin{tabular}{|c c c c|}
    \hline
    Year & Total housing units & Occupied units & Vacant units \\ [0.5ex]
    \hline
    2010 & 302,465 & 280,453 & 22,012 \\
    2011 & 304,164 & 282,480 & 21,684 \\
    2012 & 306,694 & 285,476 & 21,218 \\
    2013 & 309,205 & 288,439 & 20,766 \\
    2014 & 311,286 & 290,822 & 20,464 \\
    2015 & 315,950 & 296,633 & 19,317 \\
    2016 & 322,795 & 304,157 & 18,638 \\
    2017 & 334,739 & 314,850 & 19,889 \\
    2018 & 344,503 & 323,446 & 21,057 \\
    2019 & 354,475 & 331,836 & 22,639 \\
    2020 & 367,337 & 344,629 & 22,708 \\
    2021 & 362,809 & 337,361 & 25,448 \\
    2022 & 372,436 & 345,246 & 27,190 \\ [1ex]
    \hline
  \end{tabular}
  \caption{Housing Statistics in Seattle, Washington \cite{Census2010ACSDP5Y2010.DP04}}
\end{table}

\iffalse
\begin{table}[H]
  \centering
  \begin{tabular}{|c|l|c|l|}
    \hline
    Year & GDP & Year & GDP \\ [0.5ex]
    \hline
    1980 & 2857.3085 & 2005 & 13039.197 \\
    1981 & 3207.04125 & 2006 & 13815.583 \\
    1982 & 3343.78925 & 2007 & 14474.227 \\
    1983 & 3634.0365 &  2008 & 14769.86175 \\
    1984 & 4037.614 & 2009 & 14478.06725 \\
    1985 & 4338.9805 & 2010 & 15048.971 \\
    1986 & 4579.6325 & 2011 & 15599.7315 \\
    1987 & 4855.21625 & 2012 & 16253.97 \\
    1988 & 5236.438 & 2013 & 16880.68325 \\
    1989 & 5641.5795 & 2014 & 17608.13825 \\
    1990 & 5963.1445 & 2015 & 18295.019 \\
    1991 & 6158.12925 & 2016 & 18804.91325 \\
    1992 & 6520.32725 & 2017 & 19612.1025 \\
    1993 & 6858.5585 & 2018 & 20656.5155 \\
    1994 & 7287.2365 & 2019 & 21521.395 \\
    1995 & 7639.74925 & 2020 & 21322.9495 \\
    1996 & 8073.12175 & 2021 & 23594.03075 \\
    1997 & 8577.5525 & 2022 & 25744.10825 \\
    1998 & 9062.81675 & 2023 & 27357.842 \\
    1999 & 9631.17175 & \\ 
    2000 & 10250.952 & \\
    2001 & 10581.929 & \\
    2002 & 10929.10825 & \\
    2003 & 11456.4495 & \\
    2004 & 12217.19575 & \\ [1ex]
    \hline
  \end{tabular}
  \caption{Gross Domestic Product of the United States in Billions of Dollars, Adjusted for Inflation \cite{FREDGDP}}
\end{table}
\fi

\noindent
Due to the limited size of the dataset, we chose to use a $C$ value for our SVM closer to 1.0 to prevent overfitting or
underfitting of the data. 

\subsection{Results}

\subsection{Reflecting on the Model}

\newpage

\section{It was the Worst of Times}

\subsection{Restatement of the Problem}
\subsection{Assumptions and Justifications}
\subsection{Model Development}
\subsection{Results}
\subsection{Reflecting on the Model}

\newpage

\section{Rising from this Abyss}

\subsection{Restatement of the Problem}
\subsection{Assumptions and Justifications}
\subsection{Model Development}
\subsection{Results}
\subsection{Reflecting on the Model}

\newpage

\Urlmuskip=0mu plus 1mu\relax
\bibliographystyle{plain}
\bibliography{sources}

\end{document}
